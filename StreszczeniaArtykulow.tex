\documentclass[a4paper,10pt]{article}
\usepackage[utf-8]{inputenc}
\usepackage [T1]{fontenc}
\usepackage [english,polish]{babel}
\usepackage{polski}
\usepackage{graphics}
%\usepackage[left=2, top=1, right=2, noheadm,nofoot]{geometry}

\setlength{\textheight}{24cm}
\setlength{\textwidth}{15.92cm}
%\setlength{\footskip}{10mm}
\setlength{\oddsidemargin}{0mm}
\setlength{\evensidemargin}{0mm}
\setlength{\topmargin}{0mm}
\setlength{\headsep}{0mm}
\begin{document}

\title{E-health}
 \author{ \LARGE Dorota Wojta³ów \\ \\ \Large Informatyka Stosowana rok 4}
\date{17.11.2010}
\maketitle
\section{ Co kryje się pod pojęcie e-health}
\subsection {What is e-health (2): The death of telemedicine?}
Artykuł o tytule "What is e-health (2): The death of telemedicine" wyjaśnia stosunkowo nowe pojęcie jakim jest e-health oraz pokrótce przedstawia
różne jego aspekty. Vincenzo Della Mea, pracownik Instytutu Patologii włoskiego University of Udine, autor atykułu przedstawia e-health jako pojęcie 
ogólne opisujące wykorzystanie informatyki i komunikacji drogą elekorniczną w sektorze zdrowotnym. Według autora e-health nie ogranicza się do (jak wielu osobom
mogłoby się kojarzyć)
konsultacji medycznych drogą internetową  ale obejmuje gromadzenie, przesyłanie i odzyskiwanie danych służących do ceów klicznicznych, edukacyjnych czy nawet administracyjnych drogą elekroniczną.
Autor uważa wprowadzenie pojęcie e-health  za śmierć telemedycyny, która jako węższa dziedzina ma zostać zastąpiąna przez zdecydowanie ogólniejszy e-health.
W artykule przedstawiony jest także komercyjny aspekt wprowadzenia nowego pojęcia. E-health, podobnie jak e-commerce jest pojęciem mało precyzyjnym, a zatem
automatycznie bardziej otwartym i obiecującym.
Pierwsze pojawienie sie pojęcia e-halth datuje się na rok 1999.

\subsection{What is e-health? - Gunther Eysenbach}
Gunther Eysenbach, podobnie jak 


\end{document}
