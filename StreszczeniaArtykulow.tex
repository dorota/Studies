\documentclass[a4paper,10pt]{article}
\usepackage[utf8]{inputenc}

\usepackage [T1]{fontenc}
\usepackage [english,polish]{babel}
\usepackage{polski}
\usepackage{graphics}
%\usepackage[left=2, top=1, right=2, noheadm,nofoot]{geometry}

\setlength{\textheight}{24cm}
\setlength{\textwidth}{15.92cm}
%\setlength{\footskip}{10mm}
\setlength{\oddsidemargin}{0mm}
\setlength{\evensidemargin}{0mm}
\setlength{\topmargin}{0mm}
\setlength{\headsep}{0mm}
\begin{document}

\title{E-health}
 \author{ \LARGE Dorota Wojtałów \\ \\ \Large Informatyka Stosowana rok 4}
\date{17.11.2010}
\maketitle
\section{ Co kryje się pod pojęcie e-health}
\subsection {What is e-health (2): The death of telemedicine?}
Artykuł o tytule "What is e-health (2): The death of telemedicine" wyjaśnia stosunkowo nowe pojęcie jakim jest e-health oraz pokrótce przedstawia
różne jego aspekty. Vincenzo Della Mea, pracownik Instytutu Patologii włoskiego University of Udine, autor atykułu przedstawia e-health jako pojęcie 
ogólne opisujące wykorzystanie informatyki i komunikacji drogą elekorniczną w sektorze zdrowotnym. Według autora e-health nie ogranicza się do (jak wielu osobom
mogłoby się kojarzyć)
konsultacji medycznych drogą internetową  ale obejmuje gromadzenie, przesyłanie i odzyskiwanie danych służących do ceów klicznicznych, edukacyjnych czy nawet administracyjnych drogą elekroniczną.
Autor uważa wprowadzenie pojęcie e-health  za śmierć telemedycyny, która jako węższa dziedzina ma zostać zastąpiąna przez zdecydowanie ogólniejszy e-health.
W artykule przedstawiony jest także komercyjny aspekt wprowadzenia nowego pojęcia. E-health, podobnie jak e-commerce jest pojęciem mało precyzyjnym, a zatem
automatycznie bardziej otwartym i obiecującym.
Pierwsze pojawienie sie pojęcia e-halth datuje się na rok 1999.

\subsection{What is e-health? - Gunther Eysenbach}
Gunther Eysenbach, podobnie jak Vincenzo Della Mea zauważa obszerność pojęcia e-health. Według autora era e-healthu ma rozwiązać problemy komunikacyjne 
w 3 płaszcyznach: klient - klient, klient - szpital, szpital - szpital. Gunther w artykule podkreśla także, że e-health to nie tylko połączenie 
komputerów z medycyną ale w szerszym sensie także nowy sposób myślenia, podejście, nowa świadomość obywateli. Autor przytacza  10 znaczeń przedrostka
"e", których spełnienie nada pełną wartość słowu e-health. Należą do nich 
takie slowa jak Education - e-health ma szerzyć edukację zarówno wśród pracowników służby zdrowia, jak i klienta,
Encouragement - oznaczający tworzenie nowej 'partnerskiej' więzi między pacjentem i lekarzem,
czy Ethics - skupiający się na problemtach etycznych związanych z nowymi sposobami komunikacji i przekazywania informacji.Warto także zauważyć, że `'e-health`' nie jest pojęciem akademickim. Powstało ono naj
prawdopobniej w kręgach biznesowych i wg. autora nie powinno być używane w nauce.

\section {Rozwój e-healthu na świecie}
\subsection{\" Implementing e-Health in Developing Countries Guidance and Principles}
Dokument ten jak sam tytuł wskazuje jest bogaty we wskazówki dotyczące przeprowadzania projektów związanych z e-health, ich wdrażania w róznych państwach, problemów. Ponadto, zawiera także 
dość obszerne wyjaśnienie pojęcia e-health, kierując uwagę czytelnika na rózne aspekty. Artykuł porusza także krótko kwestie trendów w dziedzinie e-healthu.

\subsection{National Progress Report on Health 2010 - eHealth Initiative}
Jest to dokument przedstawiający działania Stanów Zjednoczonych Ameryki zmierzające do poprawy jakości usług medycznych poprzez wykorzystanie health information technolofy (HIT) i health information exchange(HIE). Składa się on z raportów, które opisują działania mające rozwiązać konkretne problemy, ich rezultaty, priorytety i braki. Raport zawiera także opinie ludzi na temat 
rezultatów prowadzonych prac. Jednym z ciekawszych, opisanych w raporcie działań jest zaangażowanie pacjentów w troskę o swoje zdrowie oraz poprawa kontaktu między
usługodawcą i usługobiorcą świadczeń medycznych. Jednym z działań zaproponowanych przez  ARRA (American Reinvestment and Recovery Act) było stworzenie narzędzi umożliwiających pacjentom dostęp 

do wiedzy  i dotyczących ich informacji medycznych (np. wyników badań), a także rozsyłanie drogą elektroniczną przypomnień np. o terminach wizyt. Raport sugeruje, że zaangażowanie pacjetnów
w troskę o zdrowie jest kluczowe w podniesieniu jakości usług medycznych. Dostęp do informacji drogą elektroniczną jest wielkim postępem w tym kierunku jednak niesie ze sobą też pewne problemy.
Jednym z nich jest poufność danych, oraz zaufanie pacjentów że zasady poufności są przestrzegane. Innym problemem jest także dostępność rozwiązań tego typu oraz ich dostosowanie dla ogółu 
społeczeństwa (szczególnie ludzi starszych, niepełnosprawnych, chorych, o różnym poziomie edukacji i w różnych państwach).   
Głównym celem jaki przyświeca powyższym działaniom jest skierowanie służby zdrowia w stronę pacjenta (umieszczenie go w centrum). Czynnikiem niezbędnym do spełnienia tego celu  jest łatwy
i stały dostęp do zrozumiałych informacji. Innym bardzo ważnym aspektem jest dostępność  nowoczesnych narzędzi wśród małych placówek i gabinetów. W tym względzie jest niestety ciągle bardzo wiele do zrobienia.
\subsection {The e-Health Development Framework in Spain - Telemedicine and Information Society, Research Division}
Artykuł ten przedstawia obecną sytuację e-Healthu w Hiszpani: bariery i siły jakie kształtują aktualny stan, problemy z którymi należy się zmierzyć, plany na przyszłość o raz porównanie z innymi państwami Europy. W pierwszzym rozdziale omawiane artykułu (nie licząc wstępu) przedstawiona jest definicja e-Healthu w rozumieniu autora. Autor ogranicza tutaj rozumeinie e-healthu do pojęcia "Health On-line" czy "Helth on the Net" wykluczając z tej definicji wszystkie aplikacje admnistracyjne, czy służące samym lekarzom. Wprowadzone jest także pojęcie Health telematics, które obrazuje wszystkie inne aplikacje wykorzystywane przez służbę zdrowia i nie korzystające z dobrodziejstw internetu.
Kolejny rozdział przedstawia początku e-Healthu w Europie. Momentem przełomowym byl grudzień 1999, kiedy to Unia Europerjska wprowadzilam program e-Europe, który miał przynieść ukazać dobrodziejstwa
internetu wszystkim obywatelom. Jednym z punktów tego planu była symulacja użycia Internetu w służbie zdrowia Health-Online, mająca na celu przede wszystkim wzrost świadomości i wiedzy medycznej
wśrdód mieszkańców. Autor porównuje także rozwój e-healthu w Europie do Stanów Zjednoczonych i Kanady zwracając uwagę na duże zainteresowanie ze strony rządu we wdrażaniu nowych technologii do medycyny.
Bardzo ważnym aspektem e-healthu, na który wskazuje autor jest dostęp do nowoczesnych rozwiązać technicznych wśród osób starszych i chorych, będących główną grupą korzystającą z usług medycznych.Drugą grupą
szczególnie zainteresowaną w dostępie do nowoczesnych rozwiązań są lekarze oraz pracownicy naukowi, którym internet służy do pogłębiania wiedzy i dokonywania nowych odkryć. Kolejnym aspektem
są elektroniczne recepty, które mogłyby zrewolucjonizować dostęp do leków.
Rozdział trzeci dostarcza szczegółowe informacje na temat sił pociągowych i barier kierujących rozwojem e-healthu. Jednym z największych czynników rozwoju jest sam rozwój internetu i techniki.
Rozwój internetu i lepszy dostęp do niego wśród mieszkańców powoduje coraz większe korzystanie z usług internetowych (w tym także medycznych). Popularyzacja tychże usług daje korzyści finansowe, 
co za tym idzie możliwości ich rozwoju, a w efekcie jeszcze większą popularyzację.Innym, bardzo istotny czynnikiem jst także polityka państwa. W Hiszpanii główną barierą hamującą rozwój
e-healthu jest słaby dostęp do internetu wśród mieszkańców. Autor zwraca także uwage na jakość informacji zamieszcanych w interncie, zaufanie społeczeństwa oraz świadomość o korzyściach jakie daje
e-health. Pod koniec arykułu, autor wyszczególnia kwestie z którymi musi się zmierzyć Hiszapnia w rozwoju sektora medycznego. Należą do nich: ackeptacja nowych metod pracy przez lekarzy, jakość
usług świadczonych droga internetową (w szczególności materiałów edukacyjnych), bezpieczeństwo danych, narzędzia usprawniające prace administracyjne, poprawa dostępu do Internetu.
\subsection {West Wirginia Health Information Technology Statewide Strategic Plan - September 2009}

\end{document}
